\chapter{Configuración en servidor remoto}

\section{Crear un usuario en Ubuntu y configurar acceso remoto por SSH}

Este es el proceso para crear un usuario, darle permisos de administrador y configurar el acceso remoto por SSH.

\subsection{Crear un usuario y añadirlo al grupo \texttt{sudo}}

\begin{figure}[h!]
    \centering
    \begin{minted}{bash}
        sudo adduser usuario
        sudo usermod -aG sudo usuario
    \end{minted}
    \caption{Comandos para crear un usuario y asignar permisos de administrador (\texttt{sudo}).}
    \label{fig:adduser-sudo}
\end{figure}

\begin{itemize}
    \item \texttt{-a}: Añade al usuario a un grupo existente sin sobrescribir otros grupos a los que ya pertenece.
    \item \texttt{-G}: Especifica los grupos secundarios a los que el usuario será añadido.
\end{itemize}

\subsection{Crear la carpeta \texttt{.ssh} para claves SSH}

Ya una vez tenemos los permisos creados para usuario. Simplemente nos cambiamos a su perfil haciendo:

\begin{figure}[h!]
    \centering
    \begin{minted}{bash}
        sudo su usuario
    \end{minted}
    \caption{Cambio a usuario.}
    \label{fig:adduser-sudo}
\end{figure}

Ahora gracias a lo ya ejecutado en la figura \ref{fig:adduser-sudo}. Como tenemos permisos de administrador, podemos
obviar el comando sudo (en caso de que este fuera necesario).

\begin{figure}[h!]
    \centering
    \begin{minted}{bash}
        mkdir /home/usuario/.ssh
    \end{minted}
    \caption{Comandos para crear un usuario y asignar permisos de administrador (\texttt{sudo}).}
    \label{fig:adduser-sudo}
\end{figure}

Ahora nos fijamos en los permisos que tiene la carpeta. Son muchos. Demasiados para este protocolo que requiere algo mas de seguridad.

\begin{figure}[h!]
    \centering
    \includegraphics[width=0.5\textwidth]{imagenes/ll_ssh.png}
    \caption[Detalle de los permisos para las carpetas o ficheros alojados en nuestro path.]{Detalle de los permisos para las carpetas o ficheros alojados en nuestro path.}
    \label{fig:ll_ssh}
\end{figure}

Aqui tenemos una ligera idea de los permisos requeridos. Luego efectuamos el cambio, ver figura \ref{fig:cambio_permisos_ssh}.

\begin{figure}[h!]
    \centering
    \includegraphics[width=0.5\textwidth]{imagenes/permisos_ssh.png}
    \caption[Permisos para lo alojado en la carpeta .ssh. Fuente: \href{https://jonasbn.github.io/til/ssh/permissions\_on\_ssh\_folder\_and\_files.html}{jonasbn.github.io}]{Permisos para lo alojado en la carpeta .ssh. Fuente: \href{https://jonasbn.github.io/til/ssh/permissions\_on\_ssh\_folder\_and\_files.html}{jonasbn.github.io}}
    \label{fig:permisos_ssh}
\end{figure}

\begin{figure}[h!]
    \centering
    \begin{minted}{bash}
        sudo chmod 700 /home/usuario/.ssh
    \end{minted}
    \caption{Cambio de permisos para la carpeta .ssh a 700.}
    \label{fig:cambio_permisos_ssh}
\end{figure}

\begin{itemize}
    \item La carpeta \texttt{.ssh} es donde se almacenan las claves SSH.
    \item El permiso \texttt{700} asegura que solo el propietario pueda leer, escribir y ejecutar.
\end{itemize}

\subsection{Configurar las claves autorizadas}

A priori tenemos un fichero con extension .pem. Este al ser la clave privada tendra un permiso 600. Que debemos asegurarnos de que sea asi.
Ademas debera pertenecer al propio usuario. Lo unico que este debe estar alojado en nuestro ordenador personal (en local). En el servidor remoto
tiene que estar con la denomicacion \texttt{authorized\_keys} la clave publica. Pero podemos dejar a modo de placeholder hasta ponerla su fichero de destino vacio.
Ademas con sus permisos correctos

\begin{figure}[h!]
    \centering
    \begin{minted}{bash}
        touch /home/usuario/.ssh/authorized_keys
        sudo chmod 600 /home/usuario/.ssh/authorized_keys
        sudo chown -R usuario:usuario /home/usuario/.ssh
    \end{minted}
    \caption{Se crea \texttt{authorized\_keys} y se le asignan los permisos correspondientes para la correcta configuración.}
    \label{fig:creamos_authorized_keys}
\end{figure}

\begin{itemize}
    \item Con el comando touch creamos el nuevo fichero \texttt{authorized\_keys}.
    \item El archivo \texttt{authorized\_keys} contiene las claves públicas permitidas para autenticarse.
    \item \texttt{chmod 600}: El archivo debe ser accesible solo para el propietario.
    \item \texttt{chown}: Cambia la propiedad de la carpeta \texttt{.ssh} y su contenido al usuario correspondiente.
\end{itemize}


\subsection{Convertir un archivo \texttt{.pem} a una clave pública}

Si tienes una clave privada en formato \texttt{.pem}, puedes generar la clave pública de esta manera:


\begin{figure}[h!]
    \centering
    \begin{minted}{bash}
        ssh-keygen -y -f authorized_keys_pem > authorized_keys
    \end{minted}
    \caption{Se crea \texttt{authorized\_keys} y se le asignan los permisos correspondientes para la correcta configuración.}
    \label{fig:creamos_authorized_keys}
\end{figure}

\subsection{Crear un archivo \texttt{authorized\_keys\_pem}}

Le ponemos una denomicacion diferente para no confundirla con la publica de destino.

\begin{figure}[h!]
    \centering
    \begin{minted}{bash}
        touch authorized_keys_pem
    \end{minted}
    \caption{Se crea \texttt{authorized\_keys} y se le asignan los permisos correspondientes para la correcta configuración.}
    \label{fig:creamos_authorized_keys}
\end{figure}

Esto crea un archivo vacío llamado \texttt{authorized\_keys\_pem}, necesario para almacenar la clave privada.
Finalmente quedara copiar el contenido de esta dentro del authorized\_keys publico que habiamos hecho anteriormente en la figura \ref{fig:creamos_authorized_keys}.

Luego abrimos el editor de texto nano (ya preinstalado) copiamos y pegamos el contenido de \texttt{authorized\_keys\_pem} en \texttt{authorized\_keys}.
Tambien podriamos simplemente haber copiado \texttt{authorized\_keys\_pem} dentro de la misma carpeta con el nombre \texttt{authorized\_keys} y despues
asignarle los permisos correspondientes como hicimos en la figura \ref{fig:creamos_authorized_keys}.

La configuracion ya quedaria lista para sentencia.

\section{Configurar VSCode}

Aqui tenemos que acceder a la siguiente ventana, apretando \texttt{ctrl+shift+p}. En caso de que no aparezca, poner en la ventana ssh y buscar dicho resultado.

\begin{figure}[h!]
    \centering
    \includegraphics[width=0.5\textwidth]{imagenes/ssh_vscode.png}
    \caption[Ventana de busqueda del entorno de VSCode]{Ventana de busqueda del entorno de VSCode}
\end{figure}

Ahora hacemos \textit{click} y vamos a la carpeta alojada en nuestro usuario del \texttt{SO} y despues incorporamos las variables
que necesita VSCode para poder conectarse via OpenSSH.

\begin{figure}[h!]
    \centering
    \includegraphics[width=0.5\textwidth]{imagenes/carpeta_local_ssh.png}
    \caption[Ventana de busqueda del entorno de VSCode]{Ventana de busqueda del entorno de VSCode}
\end{figure}

Esta seria la configuracion:

\begin{figure}[h!]
    \centering
    \includegraphics[width=0.5\textwidth]{imagenes/archivo_configuracion_ssh_vscode.png}
    \caption[Ventana de busqueda del entorno de VSCode]{Ventana de busqueda del entorno de VSCode}
\end{figure}

\begin{figure}[h!]
    \centering
    \includegraphics[width=0.5\textwidth]{imagenes/esquina_inferior_izquierda.png}
    \caption[Ventana de busqueda del entorno de VSCode]{Ventana de busqueda del entorno de VSCode}
\end{figure}

\begin{figure}[h!]
    \centering
    \includegraphics[width=0.5\textwidth]{imagenes/connect_to_host_vscode.png}
    \caption[Ventana de busqueda del entorno de VSCode]{Ventana de busqueda del entorno de VSCode}
\end{figure}

\begin{figure}[h!]
    \centering
    \includegraphics[width=0.5\textwidth]{imagenes/azure.png}
    \caption[Ventana de busqueda del entorno de VSCode]{Ventana de busqueda del entorno de VSCode}
\end{figure}

Saldra una ventana en la que tenemos que poner el SO de destino, en nuestro caso a priori deberia ser Linux (independientemente de la distribucion).

\newpage
\section{Instalacion de miniconda}

\begin{figure}[h!]
    \centering
    \begin{minted}{bash}
        mkdir $HOME
        cd $HOME
        wget https://repo.anaconda.com/miniconda/Miniconda3-latest-Linux-x86_64.sh -O miniconda.sh
        chmod +x miniconda.sh
        ./miniconda.sh
        export PATH=~/miniconda3/bin:$PATH
        source ~/.bashrc
    \end{minted}
    \caption{Se crea \texttt{authorized\_keys} y se le asignan los permisos correspondientes para la correcta configuración.}
    \label{fig:creamos_authorized_keys}
\end{figure}

\begin{figure}[h!]
    \centering
    \begin{minted}{bash}
        conda --version
        conda create -n nombre\_del\_entorno  python=3.12
    \end{minted}
    \caption{Se crea \texttt{authorized\_keys} y se le asignan los permisos correspondientes para la correcta configuración.}
    \label{fig:creamos_authorized_keys}
\end{figure}